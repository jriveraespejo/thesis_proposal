%---------------------------
% section 1
%---------------------------
\section{Topic justification and context}

The short and long term benefits of effective teaching practices can be observed throughout the literature: improvements in student achievements \citep{Rockoff_2004, Rivkin_et_al_2005, Duflo_et_al_2009, Hanushek_et_al_2012, Muralidharan_et_al_2013, Chetty_et_al_2014a, Araujo_et_al_2016}; development of executive functions \citep{Araujo_et_al_2016}, increased college attendance, higher salaries, lower possibility of premature parenthood \citep{Chetty_et_al_2014b}, among others. Similarly, the literature has shown most of the negative impacts resulting from the presence of teacher shortages\footnote{\citet{Bertoni_et_al_2020a} defined it as the context in which the teacher's supply, i.e. the number of available teachers in the system, is less than its demand. The authors further elaborate that one of the causes of these shortages is related to the applicants' lower quality or due to their faulty initial training, implying that the shortage can also be conceived as the lack of good quality teachers. In this sense, the evidence of such shortage has been more prevalent, but not decisive, with temporary teachers, as they are usually associated with inferior attributes, compared to their contracted counterparts} \citep{Duflo_et_al_2009, Muralidharan_et_al_2013, Duflo_et_al_2015, Ayala_2017, Marotta_2019} or ineffective teaching practices \citep{Hanushek_et_al_2012}.

However, while the evidence have a solid methodological support, \citet{Hanushek_et_al_2006} have indicated that some of the proxy variables used, are not consistently related to either teacher effectiveness or quality of instruction, examples of such are: out of field teaching\footnote{\citet{Medeiros_et_al_2018} defines it as teachers teaching a subject in which they are not specialized or do not have the appropriate certificate.} \citep{Ingersoll_1998, Dee_et_al_2008, Bertoni_et_al_2020a}; teaching hours \citep{Bruns_et_al_2015}; years of experience or educational degree \citep{Rockoff_2004, Rivkin_et_al_2005, Clotfelter_et_al_2006, Clotfelter_et_al_2007, Hanushek_et_al_2012}; among others.

Given the lack of consistency of the effects that arises from using such proxies, \citet{Hanushek_et_al_2012} have pointed out that the analysis of teacher effectiveness has largely turned away, from attempts to identify the teacher's specific characteristics, to focus its attention into measuring the direct relationship between them and the student outcomes\footnote{The method is known as value-added analysis, and it is based on the perspective that a good teacher is one who consistently gets higher achievement from students after other determinants of such are controlled for. For a more detailed explanation of the method refer to \citet{Scherrer_2011}.}. For that reason, considerable uncertainty is still present in the literature, regarding exactly which aspects of teachers are key for the student's learning and whether those qualities can be measured \citep{Rockoff_2004, Clotfelter_et_al_2006}.

However, because the evidence still largely supports the perception that teachers are the main driver behind the student's learning processes, one of the main points in the agenda of any educational authority should be the design of an assessment system that can attract, select, develop, and retain the most effective ones \citep{Elacqua_et_al_2018}. But first, the authority would have to define the Educational Performance Standards (EPS) that best agrees with the country's context. With the EPS establishment, the authorities can set clear expectations about what a "good" teacher should know and know to do \citep{Hincapie_et_al_2020}.

But because of the uncertainty surrounding such specific requirements, these conditions are not easy to define. Nevertheless, while the specifics are hard to determine, \citet{Hincapie_et_al_2020} has hinted that most of them can be largely grouped into two: (i) to have the disciplinary knowledge and pedagogical practices adequate to the classroom characteristics, context and teaching level, and (ii) to display such knowledge and practices in the classroom, using the appropriate material and technological resources available. 

As one can infer from the previous general conditions, and the slew evidence, knowledge is a relevant observable factor that it is consistently associated with teacher effectiveness and growth in student's achievement \citep{Santibanez_2006, Clotfelter_et_al_2006, Clotfelter_et_al_2007, Hanushek_et_al_2006, Marshall_2009, Rockoff_et_al_2011, Kane_et_al_2011, Kane_et_al_2012, Ome_2012, Metzler_et_al_2012, Kane_et_al_2013, Araujo_et_al_2016, Bietenbeck_et_al_2018, Estrada_2019}; and in that sense, its measurement should be of interest for any educational authority.

The measurement of knowledge has a myriad of available tools, and while \citet{Bertoni_et_al_2020b} had advocated for the use of multiple instruments, it is important to keep in mind that any educational department are bounded by budgetary constraints. In this setting, and compared to other instruments, valid\footnote{the extend to which a measurement tool is well-founded and accurately corresponds to the real measure \citep{Kelley_1927}} and reliable\footnote{the overall consistency of a measure under consistent conditions.} standardized tests\footnote{Assessment instrument in which the implementation, questions, scoring processes, and interpretations are consistent with a predetermined or typified way. The instrument is usually composed of questions or items that fulfill three conditions: (i) they are polytomous, i.e. they have multiple choices, (ii) the choice categories are nominal, i.e. do not present any specific order, and (iii) there is only one "correct" category or answer \citep{Rivera_2019}} stand out not only for its cost-effectiveness and a much simpler implementation \citep{Hincapie_et_al_2020}, but also because, they are one of tools with less subjective scoring processes and interpretations.

However, as no instrument is perfect, the teacher's subject knowledge scores will likely reflect measurement error \citep{Metzler_et_al_2012}. As established by \citet{Angrist_et_al_1999}, measurement error in the explanatory variable could bias the estimated coefficients. This last result implies, that evidence based on test scores could be an attenuated reflection of the true effects. On the other hand, the use of one composite value, i.e. the score, does not allow to test which specific factors -if any- leads to better or worse teacher performance, making also difficult to know which teachers should be hired or what should be done to train them \citep{Hanushek_et_al_2012}.

But beyond the use of test results as explanatory variables in modeling processes, there is one more pressing argument on why the issue of measurement error should be addressed: approximately 60\% of the Caribbean and Latin American countries use standardized test scores as part of or as a main teacher selection tool \citep{Hincapie_et_al_2020}. In this setting, devoting effort to assess the issues related to measurements errors, could help the educational authorities to understand if the scores thresholds used for the selection are appropriately set, and ultimately, to know the what kind of teachers are being integrated into the public teaching staff.

In summary, teachers are one of the main drivers behind the student achievements. However, some of the evidence supporting this claim has been based on proxy variables that are not consistently related to the quality of instruction, or methods that are not concerned with the outline of the teaching factors responsible for the student's learning. Nevertheless, while the literature still reflects considerable uncertainty on what are the "ingredients for a good teacher", a good amount of evidence has supported the disciplinary and pedagogical knowledge as relevant components of the teacher effectiveness. Finally, the literature has shown that valid and reliable standardized tests are among the best tools to assess such factors, but also have emphasized that such scores could reflect the teacher's abilities with considerable noise.

In that sense, this research plans to fill three main literature gaps. First, the researcher will use hierarchical latent variable models to obtain a noise-free score for the competencies of teachers. Second, the method would also help to obtain a dynamic multidimensional depiction of their disciplinary abilities. And lastly, the researcher will tests the real implications of the method in a data composed of repeated large standardized educational assessments from Peru. 

Concerning the first two objectives of the research, the author expects to appraise: (i) if hierarchical latent variable models can provide a general framework that could serve multiple psychometric purposes; and (ii) what are the advantages or disadvantages of using such models.

For the last objective, the author expects to shed some lights about key policy decisions related to those large evaluation processes. To mention a few: (i) do the instruments guarantee a fair assessment of minority groups with different abilities?; (ii) are we screening the most knowledgeable teachers?; (iii) what are the general characteristics of the career applicants?; (iv) what differentiate a contract teacher from a temporary one?; (v) what is the main evolution of the disciplinary knowledge of the teachers?, and, is there any identifiable divergence from such pattern?; (vi) does initial training or socioeconomic status proxy variables explain different levels of disciplinary knowledge?; (vii) what specific factor of the disciplinary knowledge is consistently related to classroom observation scores.

In this sense, the researcher believes the present thesis contributes to the literature in two aspects: (i) in a the theoretical and methodological sense, as the research is focused on offering a exhaustive description and analysis of the models; and (ii) in a more practical sense, as it helps to provide evidence on some of key policy decisions that Latin America countries are currently facing.


%---------------------------
% section 2
%---------------------------
\section{Methods}

Four measurements issues receive considerable attention in the research literature: (a) random measurement error, (b) the focus of test on particular portions of the achievement distribution, (c) cardinal versus ordinal comparisons of test scores, and (d) the multidimensionaly of educational outcomes. Not only do the test measurements issues introduce noise into the estimates of the teacher effectiveness, but they also bias upwards estimates of the variance in teacher quality \citep{Hanushek_et_al_2012}. While this was mentioned for the value-added measures it is equally valid for the standardized evaluation of teachers.


We address measurement error by correcting the estimated coefficients using a reliability ratio estimated on the basis of answers to all items on the teacher tests (see Section 5.3).


- Paragraph's main point: what method are you using?

one can improve the value-added measures if we incorporate other measures of teacher quality, such as teacher characteristics \citep{Chetty_et_al_2014a}

- Paragraph's idea 1: IRT and the focus on items

De esta forma, mientras que los modelos para respuestas dicot?micas, tales como Rasch (\citealp{Rasch1980}), de uno, dos, tres par?metros (\citealp{Lord_Nov2008}) y cuatro par?metros (\citealp{McDonald1967}), expresan la probabilidad de elegir la alternativa correcta en funci?n de la ``habilidad'' del individuo; el \textbf{Modelo de Respuesta Nominal (NRM)} y todas sus extensiones (\citealt{Bock1972}  y \citealt[cap?tulo 2]{Linden1997}), expresa la probabilidad de elegir cada alternativa de la pregunta en funci?n de la misma ``habilidad''.

A diferencia de los modelos de respuesta graduada (\citealt{Samejima1969, Samejima1972} y \citealt[cap?tulo 5]{Ham_Swam1991}), el NRM no se sustenta sobre el concepto de la dicotomizaci?n de las alternativas, que derivan en los umbrales por categor?as caracteristicos de los modelos mencionados; por el contrario, la probabilidad correspondiente a cada alternativa es modelada directamente, implementando una generalizaci?n multivariada del modelo de rasgos latentes log?stico (\citealt{Bock1972}, \citealt{Ostini2006}).



- Paragraph's idea 2: SEM and the focus on abilities




- Paragraph's idea 3: IRT and SEM equivalence (evidence)

\citep{Brown_2015}
The potential consequences of treating categorical variables as continuous variables in CFA are manifold: (1) They produce attenuated estimates of the relationships (correlations) among indicators, especially when there are floor or ceiling effects; (2) they lead to “pseudofactors” that are artifacts of item difficulty or extremeness; and (3) they produce incorrect test statistics and standard errors. ML can also produce  incorrect parameter estimates, such as in cases where marked floor or ceiling effects exist in purportedly interval-level measurement scales (i.e., because the assumption of linear relationships does not hold).


Rasch Model with SEM

1. Requires to set the loadings = 1 in all items 
(there are no evidence that different items should load differently in all sub-factors, if that happen then we can say that an item does not behave good)

2. Thresholds can be transformed into difficulty parameters. They will be from the normal ogive model.


Evidence:
It is well known that factor analysis with binary outcomes is equivalent to a two-parameter normal ogive IRT model (e.g., Ferrando & Lorenza-Sevo, 2005; Glöckner-Rist & Hoijtink, 2003; \citep{Kamata_et_al_2008, Takane_et_al_1987}.

Item difficulties have been alternatively referred to in the IRT literature as item threshold or item location parameters. In fact, item difficulty parameters are analogous to item thresholds (t) in CFA with categorical outcomes \citep{Muthen_et_al_1991}.

Item discrimination parameters are analogous to factor loadings in CFA and EFA because they represent the relationship between the latent trait and the item  responses.

Muthén (1988; Muthén et al., 1991) has shown that MIMIC models (see Chapter 7) with categorical indicators are equivalent to DIF analysis in the IRT framework (see also Meade & Lautenschlager, 2004).

Muthén (1988; Muthén et al., 1991) notes that the MIMIC framework offers several potential advantages over IRT. These include the ability to (1) use either continuous covariates (e.g., age) or categorical background  variables (e.g., gender); (2) model a direct effect of the covariate on the latent variable (in addition to direct effects of the covariate on test items); (3) readily  evaluate multidimensional models (i.e., measurement models with more than one factor); and (4) incorporate an error theory (e.g., measurement error covariances). Indeed, a general advantage of the covariance structure analysis approach is that the IRT model can be  embedded in a larger structural equation model (e.g., Lu, Thomas, & Zumbo, 2005).


De esta forma, en el contexto de una evaluaci?n estandarizada, suponemos que $n$ sujetos responden $p$ ?tems de opci?n m?ltiple eligiendo \textbf{una sola} alternativa de $m_j$ disponibles, las mismas que pueden variar de ?tem a ?tem y poseen un orden arbitrario. Entonces, el NRM define \textbf{\textit{Funciones de Respuestas de las Categor?as del ?tem}} (ICRF, acorde con \citealp{Ostini2006}) o Curvas Caracter?sticas de la Alternativas del ?tem (IOCC, acorde con \citealp{Ham_Swam1991}) de la siguiente manera:
\begin{equation}
	P_{jk}(\theta_i) = \dfrac{e^{z_{jk}(\theta_i)}}{\sum_{h=1}^{m}e^{z_{jh}(\theta_i)}} 
\end{equation}

Donde:
\begin{equation*}
z_{jk} = a_{jk}\theta_i + c_{jk} \quad \forall \quad i = 1, \dots, n; \quad j = 1, \dots, p \text{;} \quad k = 1, \dots, m_j
\end{equation*}

El par?metro $\theta_i$ representa la ``habilidad'' del individuo $i$, $a_{jk}$ corresponde al par?metro de discriminaci?n de la alternativa $k$ del ?tem $j$ y $c_{jk}$ es proporcional a la ``popularidad'' de la alternativa $k$ del ?tem $j$. El vector compuesto por los vectores $z_{j1}, z_{j2}, \dots z_{j m_j}$ es usualmente definido como el vector \textit{logit multinomial}. La presente parametrizaci?n del modelo es expresada en t?rminos del intercepto y la pendiente de las ICRFs; sin embargo, la literatura utiliza una parametrizaci?n que hace la estimaci?n computacionalmente m?s eficiente.



- Paragraph's idea 4: what can be gain from this merge

De la parametrizaci?n anterior se espera que, al igual que los modelos para respuestas dicot?micas, la ICRF de la alternativa ``correcta'' sea monot?nicamente creciente respecto a la ``habilidad'', mientras que la forma de las ICRFs de los distractores depender? de como la alternativa sea percibida por el evaluado (\citealp{Ham_Swam1991}). De este modo, se plantea estudiar la formulaci?n, supuestos, caracter?sticas y propiedades del NRM.

De manera complementaria al estudio del modelo, el presente proyecto plantea la estimaci?n de los par?metros de inter?s a trav?s de simulaciones de \textbf{Cadenas de Markov de Montecarlo (MCMC)}, perteneciente a los m?todos de inferencia bayesiana. Se elige los m?todos bayesianos debido a que: (i) elimina los problemas de no convergencia y estimaci?n impropia de los par?metros encontrados en los procedimientos de m?xima verosimilitud conjunta y/o marginal \citep{Ham_Swam_Rog1991}, (ii) bajo escenarios en los que la complejidad del modelo incrementa, el m?todo se vuelve m?s atractivo, pues usa simulaciones en vez de m?todos num?ricos; (iii) los modelos MCMC se vuelven particularmente ?tiles cuando los datos son dispersos o cuando es poco probable que la teor?a asint?tica se mantenga \citep{Fox2010}; (iv) la flexibilidad y escalabilidad de las soluciones implementadas y (v) una mayor capacidad de recuperaci?n de par?metros de inter?s, de los cuales existen muchos ejemplos (\citealp{Hsi_Proc_Hou_Teo2010}, \citealp{Tarazona2013}, entre otros).


- Paragraph's idea 5: What are the difficulties


- Paragraph's conclusion: SEM/IRT merge provides multiple benefits




%---------------------------
% section 3
%---------------------------
\section{Data}

Con respecto a los requisitos generales, para ser docente en Perú es necesario poseer el título de profesor o licenciado en educación, otorgado por una institución de formación docente acreditada en el país o en el exterior (en este último caso, el título debe ser revalidado en el Perú) 16 Además de los requisitos generales también se deben cumplir requisitos específicos, por ejemplo: a) manejar fluidamente la lengua materna de los estudiantes y conocer la cultura local para postular a vacantes de instituciones educativas pertenecientes a educación intercultural bilingüe (EIB); b) acreditar la especialización en la modalidad para postular a vacantes de instituciones educativas pertenecientes a educación básica especial (EBE); y c) se permite enseñar en inicial a los docentes con título de profesor o de licenciado en educación en la modalidad de educación básica regular (EBR) en el nivel primaria, con estudios concluidos de segunda especialidad en educación inicial y con experiencia mínima de dos (02) años lectivos en el nivel inicial.


- Paragraph's main point: What data do we have?

Finalmente, el modelo investigado ser? aplicado a un conjunto de datos reales pertenecientes al sector educativo. 


- Paragraph's idea 1: Standardized MCQ in Peru for multiple purposes

En el actual escenario de la revalorizaci?n de la carrera magisterial\footnote{Ley N? 28044, Ley General de Educaci?n}\footnote{Ley N? 29944, Ley de Reforma Magisterial}\footnote{Decreto Supremo N? 011-2012-ED, que aprueba el Reglamento de La Ley de Educaci?n}\footnote{Decreto Supremo N? 004-2013-ED, que aprueba el Reglamento de la Ley de Reforma Magisterial, y sus modificaciones}, el Ministerio de Educaci?n del Per? (MINEDU) aprob? en el a?o $2012$ e inici? la implementaci?n en el a?o $2014$ las evaluaciones a docentes con el prop?sito de: (i) evaluar las capacidades y/o competencias de los docentes nombrados en las especialidades que corresponden a su ense?anza y (ii) revalorizar las escalas salariales de los docentes nombrados. En este contexto, en el a?o $2015$, el ministerio aplic? la evaluaci?n de ``Ingreso a la Carrera Publica Magisterial y Contratacion Docente'' (en adelante \textbf{Nombramiento 2015}), la cual permiti? el ingreso de nuevos docentes a la primera de las siete escalas de la carrera magisterial.


- Paragraph's idea 2: Definition of the sample and variables

El presente proyecto opt? por implementar el modelo investigado en $40$ de los $90$ ?tems disponibles de Nombramiento $2015$, aplicados a $11826$ docentes de la especialidad de Matem?tica de la Modalidad de Educaci?n B?sica Regular Nivel Secundaria. El instrumento se encuentra dise?ado para medir un \textbf{\textit{trazo latente unidimensional}} que corresponde a las \textbf{\textit{competencias pedag?gicas y de especialidad}} que los docentes poseen. La elecci?n del modelo se sustent? en que este no solo provee informaci?n acerca de la alternativa elegida (presuntamente ``correcta''), sino tambien, permite conocer la ``popularidad'' con la que el individuo percibe el resto de categor?as disponibles, informaci?n especialmente valiosa para el an?lisis de distractores y validez te?rica de constructo de los ?tems utilizados en el instrumentos de evaluaci?n.

- Paragraph's idea 3: Composition of the exam
- Paragraph's idea 4: Selection of factors and why
- Paragraph's conclusion: The process can be performed in this data

En conclusi?n, el presente proyecto de tesis estudiar? los supuestos, propiedades y caracter?sticas del Modelo de Respuesta Nominal (NRM) e implementar? la estimaci?n de sus par?metros desde el enfoque de la  inferencia bayesiana. Entre los t?picos que adicionalmente ser?n presentados se encuentran: (i) estudios de simulaci?n que comparan la recuperaci?n de par?metros de inter?s entre el m?todo cl?sico de estimaci?n y el bayesiano y (ii) la aplicaci?n a un conjuntos de datos reales del sector educativo, acorde con lo detallado en parrafos previos.



%---------------------------
% section 3
%---------------------------
\section{Thesis objectives}

El objetivo general de la tesis consiste en estudiar la formulaci?n, supuestos, caracter?sticas y propiedades del \textbf{Modelo de Respuesta Nominal (NRM)}   en el contexto de la Teor?a de Respuesta al ?tem (IRT). Del mismo modo, se pretende realizar un estudio de simulaci?n que compare el m?todo cl?sico de estimaci?n del NRM frente a los m?todos bayesianos. Finalmente, se aplicar? el modelo descrito a un conjuntos de datos reales del sector educativo, desde el enfoque de la inferencia bayesiana. De manera espec?fica:

\begin{itemize}
\item Se realizar? una extensiva revisi?n de la literatura acerca del modelo de inter?s.
\item Se estudiar?n los supuestos, caracter?sticas y propiedades del modelo, desde la perspectiva cl?sica y bayesiana.
\item Se implementar?n m?todos de inferencia bayesiana para la estimaci?n de los par?metros de inter?s.
\item Se realizar?n estudios de simulaci?n para comprobar la capacidad de recuperaci?n de los par?metros de inter?s por parte del m?todo cl?sico y bayesiano.
\item Se aplicar? el modelo de inter?s a un conjunto de datos reales pertenecientes al sector educativo.
\end{itemize}