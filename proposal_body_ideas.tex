%---------------------------
% section 1
%---------------------------
\section{Topic justification and context}

- Paragraph's main point: Why it is esential to identify good teachers

Throughout the literature much have been emphasized about the positive impacts that the quality of the teachers has in the learning abilities of the students.

- Paragraph's idea 1: impact on student learning
  
For one, having an appropriate teacher decreases the learning gaps in the students.

Evidence on this has been found in Ecuador, at the kinderganden level \citep{Araujo_et_al_2016}, where childrens assigned to "rookie" teachers learn less on average and also better pedagogical practices are associated with learning in math, language and executive functions (working memory, capacity to pay attention, cognitive flexibility, among others).

improvements in teacher quality can raise student's test scores significantly \citep{Chetty_et_al_2014a}


Thus a teacher who is one standard deviation above the mean to the distribution of teachers in terms of quality (i.e., roughly comparing the 84th-percentile teacher to the 50th-percentile teacher) is estimated to produce marginal learning gains of 0.2–0.3 standard deviations of student achievement above the average teacher. \citep{Hanushek_et_al_2012}


De hecho, tener un buen docente tiene un efecto importante en disminuir las brechas de aprendizajes \cite{Araujo_et_al_2016; Chetty_et_al_2014a; Hanushek_et_al_2012; Rockoff_2004}, sobre todo en las escuelas que atienden a estudiantes de menor rendimiento \citep{Rivkin_et_al_2005},


La evidencia disponible sugiere que los maestros temporales aplican mayores niveles de esfuerzo y pueden tener un impacto positivo en el rendimiento de los estudiantes, cuando la renovación del contrato depende de su desempeño \citep{Duflo_et_al_2009; Muralidharan_et_al_2013}.


Using data for 5th grade, we consistently find significant returns to teacher experience in both math and reading and to licensure test scores in math achievement. \citep{Hanushek_et_al_2012}

This means that a one standard deviation increase in average teacher quality for a grade raises average student achievement in the grade by at least 0.11 standard deviations of the total test score distribution in mathematics and 0.095 standard deviations in reading. \citep{Rivkin_et_al_2005}


- Paragraph's idea 2: impact beyond the classroom and in the long run

incluso repercute en resultados de largo plazo como decisiones de educación superior, desempeño laboral o aspectos sociales, tales como el embarazo adolescente \citep{Bertoni_et_al_2020b}

we find that students assigned to high-VA teachers are more likely to attend college, earn higher salaries, and are less likely to have children as teenagers. they find A one standard deviation improvement in teacher VA in a single grade raises the probability of college attendance at age 20 by 0.82 percentage points, relative to a sample mean of 37 percent. Improvements in teacher quality also raise the quality of the colleges which students attend, as measured by the average earnings of previous graduates of that college. Students who are assigned higher VA teachers have steeper earnings trajectories in their 20s. We also find that improvements in teacher quality significantly reduce the probability of having a child while being a teenager, increase the quality of the neighborhood in which the student lives (as measured by the percentage of college graduates in that zip code) in adulthood, and raise participation rates in 401(k) retirement savings plans. \citep{Chetty_et_al_2014b}



- Paragraph's idea 3: what happens when you don't have good teachers


los docentes contratados por esta vía tienen atributos inferiores a aquellos que ganan las vacantes. Esto es muy relevante, especialmente para aquellos países que están elevando los estándares mínimos para la contratación permanente \citep{Bertoni_et_al_2020a}


... Sin embargo, cuando los contratos temporales no están sujetos a rendición de cuentas, existe evidencia de que los maestros temporales tienen una influencia negativa en los aprendizajes \citep{Ayala_2017}, especialmente en los estudiantes de bajos ingresos \citep{Marotta_2019}.


a below-average teacher leads to a similar decrease in lifetime earnings. Thus, having an effective teacher followed by an equally ineffective teacher will cancel out the gains. \citep{Hanushek_et_al_2012}


Usando una metodología de variables instrumentales, se encuentra que existe un efecto negativo de los docentes temporales sobre el desempeño académico de los estudiantes. En particular, un aumento en una desviación estándar en la proporción de docentes provisionales en secundaria en el establecimiento disminuye en 0.26 desviaciones estándar el puntaje promedio en Saber 11. El 9\% de este efecto se explica por menores competencias de los docentes, mientras que el 91\% restante se explica por otros aspectos. Adicionalmente, usando la misma variable instrumental, se encuentra que un año más de exposición a docentes provisionales disminuye el puntaje en la prueba Saber 11 en 0.15 desviaciones estándar. La evidencia encontrada junto con evidencia anecdótica parece indicar que hay un canal político mediando la contratación de docentes provisionales en los establecimientos educativos. \citep{Ayala_2017}


La evidencia a nivel internacional sobre el efecto de los docentes provisionales sobre puntaje en pruebas apunta a un efecto positivo, aunque los resultados dependen del contexto del país \citep{Duflo_et_al_2015; Muralidharan_et_al_2013; Ayala_2017}


Los autores encuentran que los estudiantes de los docentes temporales obtuvieron mejores puntajes en pruebas que los estudiantes de los docentes tradicionales y que los profesores provisionales contratados se esforzaban más que sus pares contratados por el servicio civil.
\citep{Duflo_et_al_2015}


Se observa que la proporción de docentes provisionales en el establecimiento sí afecta las
competencias básicas promedio de los docentes, entendido como el puntaje promedio en el concurso de méritos. Como se observa en la columna (2), un aumento en una desviación estándar en la proporción de docentes provisionales en el establecimiento disminuye el puntaje promedio en el concurso de méritos del establecimiento en 4 puntos. \citep{Ayala_2017}


Our findings corroborate the conclusion of the previous literature. In schools that did not receive the extra training for school committee members, students assigned to the regular teacher in ETP schools score only 0.09 standard deviations higher than students in comparison schools, and this estimate is insignificant. In contrast, students assigned to the contract teachers in the ETP schools score 0.18 standard deviations higher than students assigned to the regular teacher in the same schools (a significant difference) and 0.27 standard deviations higher than students in comparison schools. Moreover, in schools where school committees were trained, students assigned to the regular and to the extra teacher do almost
equally well, and significantly better than students in comparison schools (they score about 0.21 standard deviations higher in mathematics). \citep{Duflo_et_al_2009}


effects, this study contributes to the literature by analyzing the impact of teachers on temporary contracts and teachers who work at multiple schools at a time on student achievement. Results show that temporary teachers have a negative impact on student achievement and that such an effect is particularly stronger for low-income students. Moreover, temporary teachers are less involved in school activities and provide students with less support and feedback than teachers on permanent contracts. \citep{Duflo_et_al_2009}


Results show that temporary teachers have a negative impact on student achievement and that such an effect is particularly stronger for low-income students. \citep{Marotta_2019}


- Paragraph's conclusion: 

multiple evidence on positive impacts






- Paragraph's main point: What do we need to know about them

- Paragraph's idea 1: depends on every context, mention rubric of knowledge


nuestros gobiernos asuman compromisos serios para desarrollar e implementar una política sistémica que permita mejorar el prestigio de la profesión y la calidad de los docentes. Esta política debe incluir aspectos a lo largo de la carrera docente, como los siguientes: i) la atracción de mejores candidatos a carreras de Educación y Pedagogía; ii) la mejora de la calidad y pertinencia de la formación inicial; iii) la selección de candidatos a la profesión docente que esté basada en el mérito; iv) un marco que regule la profesión de manera clara, transparente y meritocrática; v) una política de remuneración competitiva que genere incentivos a la excelencia; y vi) una evaluación de los docentes que asegure el mejoramiento continuo \citep{Elacqua_et_al_2018}; Bruns y Luque, 2015; García et al., 2014).

Al momento de su diseño e implementación, las autoridades educativas se enfrentan a una multiplicidad de posibilidades, que incluyen la selección de instrumentos, la decisión de cómo deben o pueden implementarse y el peso que se le asignará a cada uno dentro del puntaje final, entre otros elementos. Igualmente, una vez que los procesos finalizan, debe decidirse cómo entregar la información recolectada a los docentes y, lo que es más importante aún, cómo utilizarla para asegurar la mejora de la labor pedagógica. 

Vale la pena aclarar que un paso previo al diseño del sistema de evaluación y a la elección de los instrumentos a ser utilizados tiene que ver con la definición de lo que debe saber y debe saber hacer un buen docente en cada país.

La primera pregunta que surge al analizar el tema es cómo podría la evaluación docente atraer, formar y retener en las aulas a los mejores docentes, y a través de esto mejorar el aprendizaje de los estudiantes. El gráfico 2.1 presenta la teoría de cambio, resumiendo los
mecanismos a través de los cuales esto puede llegar a suceder. 1 \citep{Hincapie_et_al_2020}





- Paragraph's idea 2: disciplinar and pedagogical knowledge are an important part

1 Para esta teoría de cambio se asume que los docentes con mayores habilidades pedagógicas y disciplinares efectivamente las despliegan en las aulas y de esta forma pueden mejorar el aprendizaje de sus estudiantes. (esto se puede probar). Para alcanzar este objetivo es necesario que todos los actores del sistema educativo tengan claro qué es lo que un docente debe saber y debe saber hacer. \citep{Hincapie_et_al_2020}


es claro que se espera que los docentes cuenten con conocimientos disciplinares y pedagógicos adecuados a su materia y grado de enseñanza, sobre sus estudiantes y su proceso de aprendizaje, sobre su contexto y sobre el uso en el aula de materiales pedagógicos y recursos tecnológicos, de tal forma que aseguren una enseñanza efectiva. \citep{Hincapie_et_al_2020}


\citep{Araujo_et_al_2016} para Ecuador, sugieren que alumnos asignados a profesores de alto desempeño dominan el contenido enseñado cerca de un 50\% más que uno asignado a un docente de bajo desempeño


teacher subject knowledge is a relevant observable factor that is part of overall teacher
quality. \citep{Metzler_et_al_2012}


The vast literature on education production functions hints at teacher knowledge as one — possibly the only — factor quite consistently associated with growth in student achievement. \citep{Metzler_et_al_2012}


In his early review of the literature, Hanushek (1986, p. 1164) concluded that “the closest thing to a consistent finding among the studies is that ‘smarter’ teachers, ones who perform well on verbal ability tests, do better in the classroom”


The results suggest that teacher subject knowledge is indeed one observable factor that is
part of what makes up as-yet unobserved teacher quality, at least in math. \citep{Metzler_et_al_2012} 




- Paragraph's idea 3: how well the teacher translate that into the student

there is a failure in the quantifiable characteristics to explain much of the variation in teacher effectiveness, possibly putting a missguided emphasis in assessments to enter the teaching career \citep{Hanushek_et_al_2012}


we find that VA models which control for a student’s prior test scores provide unbiased forecasts of teachers’ impacts on student achievement. \citep{Chetty_et_al_2014b}



- Paragraph's idea 4: socio-emotional and efficiency abilities that impact beyond the class




- Paragraph's conclusion: it is useful to know their disciplinar and pedagogical knowledge 

Todas estas políticas deben tener como objetivo desarrollar y potenciar los conocimientos disciplinares y las habilidades pedagógicas de los docentes. Las evaluaciones docentes pueden ayudar a identificar las diferencias de desempeño entre los profesores. Además, el uso adecuado de sus resultados puede otorgar la información necesaria para aprovechar al máximo sus fortalezas, buscar superar las falencias, y potenciar la excelencia en la profesión. \citep{Hincapie_et_al_2020}


Without knowing what leads to better or worse performance, it is hard to know what should be done to train teachers. It is hard to know how to hire teachers who have no observed performance. And it is hard to decide on such issues as mentoring new teachers or providing professional development. \citep{Hanushek_et_al_2012}


Considerable uncertainty remains, however, concerning exactly which aspects of teachers are important, whether those aspects can be measured, and whether that effectiveness differs by type of student. \citep{Clotfelter_et_al_2006}




- Paragraph's main point: How to identify good teachers

while teacher quality may be important, variation in teacher quality is driven by characteristics that are difficult or impossible to measure. \citep{Rockoff_2004}


A one-standard-deviation increase in teacher quality raises test scores by approximately 0.1 standard deviations in reading and math on nationally standardized distributions of achievement. \citep{Rockoff_2004} 



- Paragraph's idea 1: multiple instruments are the key

Entonces, dada la complejidad del trabajo docente, es recomendable utilizar una multiplicidad de instrumentos para identificar y seleccionar a los mejores docentes. capacidad de impactar positivamente en sus estudiantes, la evaluación docente debe basarse en instrumentos que otorguen información válida 2 y confiable 3 cualidades son fundamentales para que la autoridad educativa pueda entender cuáles son las fortalezas y las debilidades de sus docentes, y pueda llevar adelante las acciones necesarias para potenciarlas o superarlas, respectivamente

2 Cuando los resultados que surgen de su implementación permiten identificar, mediante investigaciones rigurosas, docentes altamente efectivos o que tienen un impacto en el aprendizaje de sus estudiantes
3 Cuando los resultados que se obtienen de cada docente reflejan su desempeño típico en clase y no dependen del día en particular en que la información fue recolectada o de la persona que estuvo a cargo de esa recolección. \citep{Bertoni_et_al_2020b}



Un instrumento (o un conjunto de instrumentos) de evaluación docente otorga información válida si los resultados que surgen de su implementación permiten identificar, mediante investigaciones rigurosas, docentes altamente efectivos o que tienen un impacto en el aprendizaje de sus estudiantes. Un instrumento (o un conjunto de instrumentos) otorga información confiable cuando los resultados que se obtienen de cada docente reflejan su desempeño típico en clase y no dependen del día en particular en que la información fue recolectada o de la persona que estuvo a cargo de esa recolección. Estas dos cualidades son fundamentales para que la autoridad educativa pueda entender cuáles son las fortalezas y las debilidades de sus docentes y pueda llevar adelante las acciones necesarias para potenciarlas o superarlas, respectivamente. 
Si el diseño y la implementación de las evaluaciones docentes no aseguran que los instrumentos y la información que surja de ellas sean válidos y confiables, es poco probable que los recursos invertidos en ellas tengan frutos verdaderos. Incluso si esta primera condición se cumple es importante también el uso que se les dé a los resultados que arrojen las evaluaciones. \citep{Hincapie_et_al_2020}


one can improve the value-added measures if we incorporate other measures of teacher quality, such as teacher characteristics \citep{Chetty_et_al_2014a}



- Paragraph's idea 2: standardize MCQ evaluations are good enough (at least for the purpose)

Bertoni et al (2020) - Concursos Docentes en Latinoamérica
Existe amplia evidencia de que las pruebas de conocimiento y las observaciones estandarizadas de aula son instrumentos relacionados con una mayor efectividad docente (Bruno y Strunk, 2019; Kane et al., 2011; Kane y Staiger, 2012), así como también las entrevistas por parte del director u otro funcionario (Harris y Sass, 2014; Jacob y Lefgren, 2008).

estudios encuentran que los puntajes obtenidos por los docentes en pruebas de conocimientos están asociados a mayores aprendizajes de los estudiantes \citep{Bietenbeck_et_al_2018; Clotfelter_et_al_2006; Clotfelter_et_al_2007}


Hincapie et al (2020) - Profesores a prueba
Pruebas estandarizadas a los docentes
Probablemente, la mayor ventaja de este instrumento es que, una vez su diseño asegure que las preguntas incluidas efectivamente evalúan si el docente cumple o no con los estándares de desempeño requeridos, su implementación es mucho más sencilla y menos costosa que la de los otros instrumentos arriba referenciados. Además, es el único instrumento para el cual hay evidencia causal positiva de la región que muestra cómo su implementación puede efectivamente mejorar el desempeño estudiantil (Ome, 2012; Brutti y Sánchez, 2017; Estrada, 2019).


Higher licensure test scores are associated with higher-test scores. Students assigned to teachers with higher licensure test scores apparently do better in math, but the effect is relatively modest. A one-standard-deviation increase in teacher test score implies at most a 0.017 standard deviation increase in average student math test scores and a somewhat smaller increase in reading scores. \citep{Clotfelter_et_al_2006}

higher average test scores are associated with higher math and reading achievement, with far larger effects for math than for reading. \citep{Clotfelter_et_al_2007}

The empirical analysis draws on data from the Southern and Eastern Africa Consortium
for Monitoring Educational Quality (SACMEQ). After measurement-error correction, a one-standard-deviation increase in teacher subject knowledge raises student performance by 4\% of a standard deviation. Results are robust to adding teacher fixed effects and are not driven by student or teacher sorting. Furthermore, teacher knowledge and school resources appear to be complements in student learning. This implies that teacher subject knowledge explains about 25\% of the variation in teachers’ overall effectiveness. We find that such resource based policies may be more effective in the presence of highly knowledgeable teachers. Interestingly, the coefficient on teacher subject knowledge changes only
little when teacher characteristics, such as educational attainment and experience, are
also controlled for. \citep{Bietenbeck_et_al_2018}





- Paragraph's idea 3: For now the evidence has been based on proxies, but more can be done

De acuerdo con \citep{Sutcher_et_al_2016}, los principales indicadores que han sido utilizados en la literatura son: 1) porcentaje de vacantes con dificultades para ser llenadas en los concursos docentes; 2) tamaños de clase, ya que la mayoría de los sistemas tienen límites máximos del número de estudiantes por profesor; 3) Porcentaje de docentes sin la preparación necesaria, de acuerdo a los estándares de formación inicial establecidos en la legislación; 4) Porcentaje de docentes sin experiencia; 5) Porcentaje de docentes con contratos temporales, sustitutos o con certificaciones ad-hoc para ejercer la docencia; 6) Profesores que enseñan una materia o asignatura 1 distinta a aquella en la que se especializaron o en la que obtuvieron su licencia (out of field teachers); 7) Docentes representativos de minorías étnicas, indígenas o de necesidades especiales. Cada uno de estos indicadores tiene ventajas y desventajas para aproximarse al concepto de escasez de docentes.

\citep{Bertoni_et_al_2020a}
En segundo lugar, para Brasil, Chile y Ecuador, se consideran indicadores de idoneidad de los profesores. Según Medeiros et al., (2018), la falta de idoneidad (out-of-field-teaching, en inglés), ocurre cuando el docente enseña una materia en la que no tiene la especialidad y/o el certificado correspondiente

Con respecto a los requisitos generales, para ser docente en Perú es necesario poseer el título de profesor o licenciado en educación, otorgado por una institución de formación docente acreditada en el país o en el exterior (en este último caso, el título debe ser revalidado en el Perú) 16 Además de los requisitos generales también se deben cumplir requisitos específicos, por ejemplo: a) manejar fluidamente la lengua materna de los estudiantes y conocer la cultura local para postular a vacantes de instituciones educativas pertenecientes a educación intercultural bilingüe (EIB); b) acreditar la especialización en la modalidad para postular a vacantes de instituciones educativas pertenecientes a educación básica especial (EBE); y c) se permite enseñar en inicial a los docentes con título de profesor o de licenciado en educación en la modalidad de educación básica regular (EBR) en el nivel primaria, con estudios concluidos de segunda especialidad en educación inicial y con experiencia mínima de dos (02) años lectivos en el nivel inicial.

Sin embargo se puede tener una medicion de idoneidad mas precisa, si se conocen los instrumentos. Solo usan el puntaje para medir un proxy de idoneidad


últimas décadas del siglo XX. Además, las habilidades pedagógicas y los conocimientos disciplinares de los docentes en la región están por debajo de lo que sugieren los estándares internacionales \citep{Hincapie_et_al_2020}. Por ejemplo, los docentes de la región dedican 20\% menos del tiempo efectivo en clase de lo que estos recomiendan. Es decir, en América Latina, las diferencias en tiempos de enseñanza efectiva de los docentes implican que los estudiantes de la región reciben en promedio un día menos de clase a la semana \citep{Bruns_et_al_2015}. De manera similar, en lo que respecta a los temas disciplinares, pruebas en Perú, Chile y México y estudios internacionales aplicados a los propios docentes indican que sus conocimientos de matemáticas son insatisfactorios \citep{Elacqua_et_al_2018}.


First, neither a graduate degree nor additional years of experience past the initial year or two translate into significantly higher instructional effectiveness. Second, descriptions of unequal access to quality teachers as measured by experience, education, or other quantifiable characteristics fail to portray accurately any actual differences in the quality of instruction by student demographics, community characteristics, and specific schools \citep{Hanushek_et_al_2012}


The analysis of teacher effectiveness has largely turned away from attempts to identify
specific characteristics of teachers. Instead attention has focused directly on the relationship between teachers and student outcomes. This outcome-based perspective, now commonly called value-added analysis, takes the perspective that a good teacher is simply one
who consistently gets higher achievement from students (after controlling for other determinants of student achievement such as family influences or prior teachers). \citep{Hanushek_et_al_2012}


we find little evidence that any observable teacher characteristic, save experience, explains any of this variation. \citep{Clotfelter_et_al_2006}


Ultimately, two characteristics – teacher experience and licensure test scores – emerge as robust determinants of test scores for fifth grade students. Compared to students assigned to
teachers with no prior experience, students assigned to highly experienced teachers attain standardized reading and math test scores roughly one-tenth of a standard deviation higher in math and slightly less than a tenth of a standard deviation in reading. About half of this gain occurs for the first one or two years of teaching. Students assigned to teachers with higher licensure test scores apparently do better in math, but the effect is relatively modest. A one-standard-deviation increase in teacher test score implies at most a 0.017 standard deviation increase in average student math test scores and a somewhat smaller increase in reading scores. \citep{Clotfelter_et_al_2006}


The most surprising result is the consistently negative effect of a master’s degree on student achievement. The coefficients suggest that, all else constant, teachers with master’s degrees are less effective than those without. \citep{Clotfelter_et_al_2006}


We conclude that a teacher’s experience, test scores and regular licensure all have positive effects on student achievement, with larger effects for math than for reading. Taken together the various teacher credentials exhibit quite large effects on math achievement, whether compared to the effects of changes in class size or to the socio-economic characteristics of students. \citep{Clotfelter_et_al_2007}


As expected, we find clear evidence that teachers with more experience are more effective in raising student achievement than those with less experience. Though the positive results by years of teacher experience are clear and robust to various model specifications, the thorny issue remains of whether the rising returns to experience reflect improvement with experience or differentially higher attrition of the less effective teachers \citep{Rockoff_2004}. 


Having a graduate degree exerts no statistically significant effect on student achievement and in some cases the coefficient is negative. Thus, the higher pay for graduate degrees would appear to be money that is not well spent, except to the extent that the option of getting a master’s degree keeps effective experienced teachers in the profession. \citep{Clotfelter_et_al_2007}


teachers have powerful effects on reading and mathematics achievement, though little of the variation in teacher quality is explained by observable characteristics such as education or experience. The results suggest that the effects of a costly ten student reduction in class size are smaller than the benefit of moving one standard deviation up the teacher quality
distribution, highlighting the importance of teacher effectiveness in the determination
of school quality. \citep{Rivkin_et_al_2005}

Consistent with prior findings, there is no evidence that a master’s degree raises teacher effectiveness. In addition, experience is not significantly related to achievement following the initial years in the profession. \citep{Rivkin_et_al_2005}

I also find evidence that teaching experience significantly raises student test scores, particularly in reading subject areas. Reading test scores differ by approximately 0.17 standard deviations on average between beginning teachers and teachers with ten or more years of experience. Evidence of gains from experience for math subjects is weaker. The first two years of teaching experience appear to raise scores significantly in math computation (about 0.1 standard deviations). However, subsequent years of experience appear to lower test scores, though standard errors are too large to conclude anything definitive about this latter trend. \citep{Rockoff_2004}


easily-observed teacher characteristics, such as education, gender, and teaching experience (except for the first few years), are not consistently related to teacher effectiveness \citep{Hanushek_et_al_2006}.


At least within our sample of Peruvian teachers, there is no indication that the effect
of teacher subject knowledge levels out at higher knowledge levels. \citep{Metzler_et_al_2012}



- Paragraph's idea 4: Selection has been based on scores

En el contexto de un proceso de evaluaci?n, muchas veces el individuo se ve enfrentado a una prueba ``estandarizada''; es decir, una evaluaci?n dise?ada de tal manera que las preguntas, las condiciones para ser administrada, los procedimientos de calificaci?n e interpretaciones son consistentes con una manera predeterminada o tipificada. En este contexto, el individuo es expuesto a un instrumento de evaluaci?n cuyos ?tems usualmente cumplen con las siguientes tres caracter?sticas: (i) preguntas de opci?n m?ltiple o polit?micas, (ii) preguntas que exhiben categor?as nominales, sin un ordenamiento espec?fico y (iii) una respuesta ``correcta''; tal y como se observa en la \textbf{Figura \ref{fig:mcq_nominal}}

 (measurement error problem)


Four measurements issues receive considerable attention in the research literature: (a) random measurement error, (b) the focus of test on particular portions of the achievement distribution, (c) cardinal versus ordinal comparisons of test scores, and (d) the multidimensionaly of educational outcomes. Not only do the test measurements issues introduce noise into the estimates of the teacher effectiveness, but they also bias upwards estimates of the variance in teacher quality \citep{Hanushek_et_al_2012}. While this was mentioned for the value-added measures it is equally valid for the standardized evaluation of teachers.


Other analyses have emphasized the importance of measurement error in using test outcome data (e.g., Kane & Staiger 2002, McCaffrey et al. 2009).


Like all test scores, teacher subject knowledge in our data is likely measured with error. Measurement error in the explanatory variable might lead to an attenuation bias, which is aggravated in the student fixed-effects model (\citep{Angrist_et_al_1999}, Section 4). We address measurement error by correcting the estimated coefficients using a reliability ratio estimated on the basis of answers to all items on the teacher tests (see Section 5.3).


As is well known, measurement error in the explanatory variable may lead to a downward bias in the estimated coefficient, and this bias may be aggravated in the student fixed-effects models \citep{Angrist_et_al_1999}


The only teacher trait consistently associated with gains in student performance is teacher cognitive skills as measured by achievement tests (\citep{Hanushek_et_al_2006; Rockoff_et_al_2011}). In the context of developing countries, several studies have found positive correlations between teacher test scores and student achievement; see, for example, \citep{Santibanez_2006} for Mexico, \citep{Marshall_2009} for Guatemala, and Behrman et al. (2008) for Pakistan.


We find that teacher subject knowledge exerts a statistically and quantitatively significant impact on student achievement. After measurement-error correction, one standard deviation in subject-specific teacher achievement increases student achievement by about 9\% of a standard deviation in math. Effects in reading are significantly smaller and mostly not significantly \citep{Metzler_et_al_2012}


The available measure may proxy only poorly for the concept of teachers' subject knowledge, as in most existing studies, the examined skill is not subject-specific. Furthermore, any specific test will measure teacher subject knowledge only with considerable noise. \citep{Metzler_et_al_2012}



- Paragraph's conclusion: 

Second, it would be valuable to develop richer measures of teacher quality which go beyond the mean test score impacts that we analyzed here. \citep{Chetty_et_al_2014a}

standardized evaluation help to know the knowledge of the teacher

\citep{Hincapie_et_al_2020}
Las falencias en la calidad de la formación inicial y las características de los interesados
en ingresar a la profesión docente en la región (reseñadas por \citep{Elacqua_et_al_2018}), sumadas a la dificultad que tienen los ministerios de educación para remover del cargo a docentes con bajos niveles de desempeño, convierten a las evaluaciones de ingreso a la carrera en un elemento esencial para identificar mejor las características y capacidades de los futuros docentes. En ese sentido, continuar con los procesos de mejora y de implementación adecuada de esta evaluación podría traer beneficios en la calidad de la fuerza docente en la región. Aunque la evidencia aún es escasa, estudios para Colombia (Ome, 2012; Brutti y Sánchez, 2017) y México (Estrada, 2019) sugieren que estos sistemas de selección (con evaluaciones para ingresar a la carrera) están teniendo algunos impactos positivos en la calidad educativa de los países que los implementan.

Without knowing what leads to better or worse performance, it is hard to know what should be done to train teachers or how to / which to hire them \citep{Hanushek_et_al_2012}





- Paragraph's main point: How the results are used

La segunda característica necesaria dentro de la teoría de cambio (reflejada en el segundo punto del primer círculo de la cadena) es el uso que se les da a los resultados de la evaluación. \citep{Hincapie_et_al_2020}


- Paragraph's idea 1: how they are used

la Organización para la Cooperación y el Desarrollo Económicos (OCDE) este uso puede tener dos objetivos: i) mejorar las prácticas y las habilidades pedagógicas y/o disciplinares a partir del diagnóstico y la vinculación a programas de desarrollo profesional diseñados para superar los resultados; ii) mejorar la composición y la motivación de la fuerza docente por medio del otorgamiento de bonificaciones, reconocimientos especiales o ascensos para aquellos docentes con resultados excelentes o excluir al docente del sistema —o al menos retirarlo de las aulas— en los casos en que muestre de manera consistente que no cumple con las condiciones requeridas por la profesión (OCDE, 2009 y 2013).
El problema que surge en relación con estos dos objetivos (plasmados en el segundo eslabón del gráfico 2.1) es que son difíciles de lograr con una única herramienta de evaluación.
Para poder detectar los aspectos por mejorar de las prácticas y el conocimiento pedagógico y disciplinar, y reconocer la excelencia docente, es necesario que los docentes estén completamente abiertos a revelar sus prácticas y logros y dispuestos a compartirlos con las autoridades \citep{Hincapie_et_al_2020}


- Paragraph's idea 2: evidence about impacts on the use of the results


- Paragraph's conclusion: results can have serious impacts into multiple facets




- Closing thoughts

en este documento tratamos de responder a dos preguntas fundamentales: ¿cómo identificamos y seleccionamos a los mejores docentes? y ¿cómo los asignamos a las escuelas de una manera eficiente y equitativa? \citep{Bertoni_et_al_2020b}





The proposal plans to fill the gap of the rest of the literature by developing a multidimensional measure of the disciplinary abilities of the teachers, free of measurement error, using actual results from large standardized test in Peru. To later on shed some light about some key policy decisions: (i) can a factor analysis of the test offers the same or more advantages as IRT models, (ii) does the evolution of teachers knowledge have been increasing since the actual change of the law, (iii) are the scores threshold used in the selection processes work as they should (meritocratic), (iv) what are the characteristics of the contracted vs the temporary teachers, that will provide a diagnostic of what is going in to the system and what kind of training will they need, (v) are we evaluating in a fair way diverse minority groups (invidents, language, invariance measurement), (vi) how related are the knowledge with their development in class (more related to the delivery, check how subjective is that score?), (vii) how much the initial training (ley de la que vienen) and socio-economic status are responsible for the level of disciplinary knowledge (using proxies, UBIGEO, mapearlo con NSE estimados por zonas del INEI), (viii) the level of the applicants in general




%---------------------------
% section 2
%---------------------------
\section{Methods}

- Paragraph's main point: what method are you using?


- Paragraph's idea 1: IRT and the focus on items

De esta forma, mientras que los modelos para respuestas dicot?micas, tales como Rasch (\citealp{Rasch1980}), de uno, dos, tres par?metros (\citealp{Lord_Nov2008}) y cuatro par?metros (\citealp{McDonald1967}), expresan la probabilidad de elegir la alternativa correcta en funci?n de la ``habilidad'' del individuo; el \textbf{Modelo de Respuesta Nominal (NRM)} y todas sus extensiones (\citealt{Bock1972}  y \citealt[cap?tulo 2]{Linden1997}), expresa la probabilidad de elegir cada alternativa de la pregunta en funci?n de la misma ``habilidad''.

A diferencia de los modelos de respuesta graduada (\citealt{Samejima1969, Samejima1972} y \citealt[cap?tulo 5]{Ham_Swam1991}), el NRM no se sustenta sobre el concepto de la dicotomizaci?n de las alternativas, que derivan en los umbrales por categor?as caracteristicos de los modelos mencionados; por el contrario, la probabilidad correspondiente a cada alternativa es modelada directamente, implementando una generalizaci?n multivariada del modelo de rasgos latentes log?stico (\citealt{Bock1972}, \citealt{Ostini2006}).



- Paragraph's idea 2: SEM and the focus on abilities




- Paragraph's idea 3: IRT and SEM equivalence (evidence)

\citep{Brown_2015}
The potential consequences of treating categorical variables as continuous variables in CFA are manifold: (1) They produce attenuated estimates of the relationships (correlations) among indicators, especially when there are floor or ceiling effects; (2) they lead to “pseudofactors” that are artifacts of item difficulty or extremeness; and (3) they produce incorrect test statistics and standard errors. ML can also produce  incorrect parameter estimates, such as in cases where marked floor or ceiling effects exist in purportedly interval-level measurement scales (i.e., because the assumption of linear relationships does not hold).


Rasch Model with SEM

1. Requires to set the loadings = 1 in all items 
(there are no evidence that different items should load differently in all sub-factors, if that happen then we can say that an item does not behave good)

2. Thresholds can be transformed into difficulty parameters. They will be from the normal ogive model.


Evidence:
It is well known that factor analysis with binary outcomes is equivalent to a two-parameter normal ogive IRT model (e.g., Ferrando & Lorenza-Sevo, 2005; Glöckner-Rist & Hoijtink, 2003; \citep{Kamata_et_al_2008; Takane_et_al_1987}.

Item difficulties have been alternatively referred to in the IRT literature as item threshold or item location parameters. In fact, item difficulty parameters are analogous to item thresholds (t) in CFA with categorical outcomes \citep{Muthen_et_al_1991}.

Item discrimination parameters are analogous to factor loadings in CFA and EFA because they represent the relationship between the latent trait and the item  responses.

Muthén (1988; Muthén et al., 1991) has shown that MIMIC models (see Chapter 7) with categorical indicators are equivalent to DIF analysis in the IRT framework (see also Meade & Lautenschlager, 2004).

Muthén (1988; Muthén et al., 1991) notes that the MIMIC framework offers several potential advantages over IRT. These include the ability to (1) use either continuous covariates (e.g., age) or categorical background  variables (e.g., gender); (2) model a direct effect of the covariate on the latent variable (in addition to direct effects of the covariate on test items); (3) readily  evaluate multidimensional models (i.e., measurement models with more than one factor); and (4) incorporate an error theory (e.g., measurement error covariances). Indeed, a general advantage of the covariance structure analysis approach is that the IRT model can be  embedded in a larger structural equation model (e.g., Lu, Thomas, & Zumbo, 2005).


De esta forma, en el contexto de una evaluaci?n estandarizada, suponemos que $n$ sujetos responden $p$ ?tems de opci?n m?ltiple eligiendo \textbf{una sola} alternativa de $m_j$ disponibles, las mismas que pueden variar de ?tem a ?tem y poseen un orden arbitrario. Entonces, el NRM define \textbf{\textit{Funciones de Respuestas de las Categor?as del ?tem}} (ICRF, acorde con \citealp{Ostini2006}) o Curvas Caracter?sticas de la Alternativas del ?tem (IOCC, acorde con \citealp{Ham_Swam1991}) de la siguiente manera:
\begin{equation}
	P_{jk}(\theta_i) = \dfrac{e^{z_{jk}(\theta_i)}}{\sum_{h=1}^{m}e^{z_{jh}(\theta_i)}} 
\end{equation}

Donde:
\begin{equation*}
z_{jk} = a_{jk}\theta_i + c_{jk} \quad \forall \quad i = 1, \dots, n; \quad j = 1, \dots, p \text{;} \quad k = 1, \dots, m_j
\end{equation*}

El par?metro $\theta_i$ representa la ``habilidad'' del individuo $i$, $a_{jk}$ corresponde al par?metro de discriminaci?n de la alternativa $k$ del ?tem $j$ y $c_{jk}$ es proporcional a la ``popularidad'' de la alternativa $k$ del ?tem $j$. El vector compuesto por los vectores $z_{j1}, z_{j2}, \dots z_{j m_j}$ es usualmente definido como el vector \textit{logit multinomial}. La presente parametrizaci?n del modelo es expresada en t?rminos del intercepto y la pendiente de las ICRFs; sin embargo, la literatura utiliza una parametrizaci?n que hace la estimaci?n computacionalmente m?s eficiente.



- Paragraph's idea 4: what can be gain from this merge

De la parametrizaci?n anterior se espera que, al igual que los modelos para respuestas dicot?micas, la ICRF de la alternativa ``correcta'' sea monot?nicamente creciente respecto a la ``habilidad'', mientras que la forma de las ICRFs de los distractores depender? de como la alternativa sea percibida por el evaluado (\citealp{Ham_Swam1991}). De este modo, se plantea estudiar la formulaci?n, supuestos, caracter?sticas y propiedades del NRM.

De manera complementaria al estudio del modelo, el presente proyecto plantea la estimaci?n de los par?metros de inter?s a trav?s de simulaciones de \textbf{Cadenas de Markov de Montecarlo (MCMC)}, perteneciente a los m?todos de inferencia bayesiana. Se elige los m?todos bayesianos debido a que: (i) elimina los problemas de no convergencia y estimaci?n impropia de los par?metros encontrados en los procedimientos de m?xima verosimilitud conjunta y/o marginal \citep{Ham_Swam_Rog1991}, (ii) bajo escenarios en los que la complejidad del modelo incrementa, el m?todo se vuelve m?s atractivo, pues usa simulaciones en vez de m?todos num?ricos; (iii) los modelos MCMC se vuelven particularmente ?tiles cuando los datos son dispersos o cuando es poco probable que la teor?a asint?tica se mantenga \citep{Fox2010}; (iv) la flexibilidad y escalabilidad de las soluciones implementadas y (v) una mayor capacidad de recuperaci?n de par?metros de inter?s, de los cuales existen muchos ejemplos (\citealp{Hsi_Proc_Hou_Teo2010}, \citealp{Tarazona2013}, entre otros).


- Paragraph's idea 5: What are the difficulties


- Paragraph's conclusion: SEM/IRT merge provides multiple benefits





- Paragraph's main point: What data do we have?

Finalmente, el modelo investigado ser? aplicado a un conjunto de datos reales pertenecientes al sector educativo. 


- Paragraph's idea 1: Standardized MCQ in Peru for multiple purposes

En el actual escenario de la revalorizaci?n de la carrera magisterial\footnote{Ley N? 28044, Ley General de Educaci?n}\footnote{Ley N? 29944, Ley de Reforma Magisterial}\footnote{Decreto Supremo N? 011-2012-ED, que aprueba el Reglamento de La Ley de Educaci?n}\footnote{Decreto Supremo N? 004-2013-ED, que aprueba el Reglamento de la Ley de Reforma Magisterial, y sus modificaciones}, el Ministerio de Educaci?n del Per? (MINEDU) aprob? en el a?o $2012$ e inici? la implementaci?n en el a?o $2014$ las evaluaciones a docentes con el prop?sito de: (i) evaluar las capacidades y/o competencias de los docentes nombrados en las especialidades que corresponden a su ense?anza y (ii) revalorizar las escalas salariales de los docentes nombrados. En este contexto, en el a?o $2015$, el ministerio aplic? la evaluaci?n de ``Ingreso a la Carrera Publica Magisterial y Contratacion Docente'' (en adelante \textbf{Nombramiento 2015}), la cual permiti? el ingreso de nuevos docentes a la primera de las siete escalas de la carrera magisterial.


- Paragraph's idea 2: Definition of the sample and variables

El presente proyecto opt? por implementar el modelo investigado en $40$ de los $90$ ?tems disponibles de Nombramiento $2015$, aplicados a $11826$ docentes de la especialidad de Matem?tica de la Modalidad de Educaci?n B?sica Regular Nivel Secundaria. El instrumento se encuentra dise?ado para medir un \textbf{\textit{trazo latente unidimensional}} que corresponde a las \textbf{\textit{competencias pedag?gicas y de especialidad}} que los docentes poseen. La elecci?n del modelo se sustent? en que este no solo provee informaci?n acerca de la alternativa elegida (presuntamente ``correcta''), sino tambien, permite conocer la ``popularidad'' con la que el individuo percibe el resto de categor?as disponibles, informaci?n especialmente valiosa para el an?lisis de distractores y validez te?rica de constructo de los ?tems utilizados en el instrumentos de evaluaci?n.

- Paragraph's idea 3: Composition of the exam
- Paragraph's idea 4: Selection of factors and why
- Paragraph's conclusion: The process can be performed in this data

En conclusi?n, el presente proyecto de tesis estudiar? los supuestos, propiedades y caracter?sticas del Modelo de Respuesta Nominal (NRM) e implementar? la estimaci?n de sus par?metros desde el enfoque de la  inferencia bayesiana. Entre los t?picos que adicionalmente ser?n presentados se encuentran: (i) estudios de simulaci?n que comparan la recuperaci?n de par?metros de inter?s entre el m?todo cl?sico de estimaci?n y el bayesiano y (ii) la aplicaci?n a un conjuntos de datos reales del sector educativo, acorde con lo detallado en parrafos previos.



%---------------------------
% section 3
%---------------------------
\section{Thesis objectives}

El objetivo general de la tesis consiste en estudiar la formulaci?n, supuestos, caracter?sticas y propiedades del \textbf{Modelo de Respuesta Nominal (NRM)}   en el contexto de la Teor?a de Respuesta al ?tem (IRT). Del mismo modo, se pretende realizar un estudio de simulaci?n que compare el m?todo cl?sico de estimaci?n del NRM frente a los m?todos bayesianos. Finalmente, se aplicar? el modelo descrito a un conjuntos de datos reales del sector educativo, desde el enfoque de la inferencia bayesiana. De manera espec?fica:

\begin{itemize}
\item Se realizar? una extensiva revisi?n de la literatura acerca del modelo de inter?s.
\item Se estudiar?n los supuestos, caracter?sticas y propiedades del modelo, desde la perspectiva cl?sica y bayesiana.
\item Se implementar?n m?todos de inferencia bayesiana para la estimaci?n de los par?metros de inter?s.
\item Se realizar?n estudios de simulaci?n para comprobar la capacidad de recuperaci?n de los par?metros de inter?s por parte del m?todo cl?sico y bayesiano.
\item Se aplicar? el modelo de inter?s a un conjunto de datos reales pertenecientes al sector educativo.
\end{itemize}